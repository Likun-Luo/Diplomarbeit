\chapter{Results and Discussions}
\label{chap:results}

In the following, the results obtained from the finite element simulation using different setups are discussed and compared to the outer pressure field measurement.

\section{Comparison of simulation and measurement results}



\begin{figure}[H]
	\centering
	\begin{subfigure}[b]{0.49\textwidth}
		\centering
		\includegraphics{fig/chap5/initial_model/third_octave_over_height/125_Hz.png}
		\caption{125 Hz}
	\end{subfigure}
	\begin{subfigure}[b]{0.49\textwidth}
		\centering
		\includegraphics{fig/chap5/initial_model/third_octave_over_height/400_Hz.png}
		\caption{400 Hz}
	\end{subfigure}
	\begin{subfigure}[b]{0.49\textwidth}
		\centering
		\includegraphics{fig/chap5/initial_model/third_octave_over_height/1250_Hz.png}
		\caption{1250 Hz}
	\end{subfigure}
	\begin{subfigure}[b]{0.49\textwidth}
		\centering
		\includegraphics{fig/chap5/initial_model/third_octave_over_height/2000_Hz.png}
		\caption{2000 Hz}
	\end{subfigure}
	\caption{Sound distribution at measurement position a, comparison between predictions obtained from the initial finite element model with the measurements. A-weighted SPL in one-third octave bands, dBA ref 20 $\mu$Pa.}
	\label{fig:third_octave_over_height}
\end{figure}

In fig. \ref{fig:third_octave_over_height}, the A-weighted sound pressure levels at measurement position a (10 cm away from vehicle) along the car body height in example one-third octave bands are shown. At each microphone position, the simulation result is compared with the measurement value. One can see that for 125 Hz and 400 Hz, the simulation fits the measured curves very well. Greater difference between the simulation and the measurement occurs for 1250 Hz and 2000 Hz, but the measured shape is still well approximated for both frequency bands. Other bands show similar results and the measured sound pressure level trends are captured well by the model.

\begin{figure}[H]
	\centering
	\begin{subfigure}[b]{0.49\textwidth}
		\centering
		\includegraphics{fig/chap5/initial_model/overall_SPL/all_pos.png}
		%\caption{2000 Hz}
	\end{subfigure}
	\begin{subfigure}[b]{0.49\textwidth}
		\centering
		\includegraphics{fig/chap5/initial_model/overall_SPL/deviation.png}
		%\caption{2000 Hz}
	\end{subfigure}
	\caption{Comparison of overall SPL in dBA ref 20 $\mu$Pa. Left: overall SPL as a function of height at various measurement positions (dashed curves: measurement; solid curves: simulation) Right: deviation between simulation and measurement results}
	\label{fig:overall_SPL}
\end{figure}

The overall sound pressure level is a simple and direct parameter to quantify the sound level. For comparison, the A-weighted sound pressure levels of each one-third octave band from 100 Hz to 2000 Hz are added up. Fig. \ref{fig:overall_SPL} shows the overall A-weighted sound pressure level at three different measurement positions and the deviation between the simulated and measured results. As can be seen from the results, the predictions of overall sound pressure levels at the three distances from car body agree well with the measurement. For all three measurement positions, the maximum deviation between the simulation and measurement results occurs at 0.5 m, which is at the height of the bogie. At these locations, the simulation results are overestimated by 1.5 dB to 3 dB. In general, the approximation is getting better with increasing height. For area above 2m, the difference between the simulation and the measurement in terms of overall sound pressure is within 1 dB.

\begin{figure}[H]
	\centering
	\begin{subfigure}[b]{0.49\textwidth}
		\centering
		\includegraphics{fig/chap5/initial_model/freq_spectrum/pos_10cm_0pt5m.png}
		\caption{0.5 m}
	\end{subfigure}
	\begin{subfigure}[b]{0.49\textwidth}
		\centering
		\includegraphics{fig/chap5/initial_model/freq_spectrum/pos_10cm_1pt5m.png}
		\caption{1.5 m}
	\end{subfigure}
\end{figure}
\begin{figure}[H] \ContinuedFloat
	\begin{subfigure}[b]{0.49\textwidth}
		\centering
		\includegraphics{fig/chap5/initial_model/freq_spectrum/pos_10cm_2pt5m.png}
		\caption{2.5 m}
	\end{subfigure}
	\begin{subfigure}[b]{0.49\textwidth}
		\centering
		\includegraphics{fig/chap5/initial_model/freq_spectrum/pos_10cm_4pt5m.png}
		\caption{4.5 m}
	\end{subfigure}
	\caption{1/3-octave spectra of the A-weighted SPL for simulation and measurement data at selected microphone positions}
	\label{fig:freq_spectrum}
\end{figure}

Fig. \ref{fig:freq_spectrum} shows the spectra of the A-weighted sound pressure level in one third octave bands. The SPL spectra are evaluated at the measurement position a (10 cm away from vehicle) at different microphone locations. As can be seen from the results, the simulation fits the measured spectra well, the measured trend is well captured. Again, larger deviation between the simulated and measured results occurs at 0.5 m above ground and the approximation is getting better with increasing height.

In order to have an idea of the model accuracy of each 1/3-octave band, the mean relative error between the simulation result and the measurement for each 1/3-octave band over all 27 microphone positions will be computed. To compute the mean relative error, the sound pressure level is first converted back to linear scaling

\begin{equation}
	p(\text{Pa}) = p_0 \cdot 10^{\frac{\text{SPL}}{20}} = 2\cdot10^{-5}\,\text{Pa} \cdot 10^{\frac{\text{SPL}}{20}}\text{.}
\end{equation}

The mean relative error (MRE) is then defined as

\begin{equation}
	\text{MRE}(p_{\text{measured}}, p_{\text{simulation}}) = \frac{1}{N} \sum_{i=0}^{N - 1} \frac{|p_{\text{simulation,}i} - p_{\text{measured,}i}|}{|p_{\text{measured,}i}|}
\end{equation}

The mean relative error can also be converted to decibel using the relation

\begin{equation}
	\text{MRE(dB)} = 20\cdot\log_{10}(1 + \text{MRE})\text{.}
\end{equation}

\begin{figure}[H]
	\centering
	\includegraphics{fig/chap5/initial_model/freq_spectrum/average_gap.png}
	\caption{Mean relative error between simulation result and measurement for each 1/3-octave band over all 27 microphone positions}
	\label{fig:gap_freq_spectrum}
\end{figure}

\begin{table}[H]
	\caption{Mean relative error of 1/3-octave frequency spectrum over all microphone positions}
	\begin{tabular}{c|cccccccccccccc}
		Freq (Hz)           & 100  & 125  & 160  & 200  & 250  & 315  & 400  & 500  & 630  & 800  & 1000 & 1250 & 1600 & 2000 \\ \hline
		MRE (dB) & 2.34 & 0.55 & 3.06 & 3.51 & 1.34 & 1.82 & 1.06 & 1.67 & 1.53 & 1.97 & 3.36 & 2.15 & 2.45 & 2.65
	\end{tabular}
	\label{tab:MRE_spectra}
\end{table}

The mean relative error in terms of the sound pressure levels over all 27 microphone positions at each frequency band between the predictions and the measurements are shown in fig. \ref{fig:gap_freq_spectrum} and in tab. \ref{tab:MRE_spectra}. As can be seen from the results, the best approximation occurs for 125 Hz band which has the smallest mean relative error among all 1/3 octave bands. Three frequency bands (160, 200 and 1000 Hz) have an error over 3 dB, and the maximum error is limited by 3.5 dB. Good approximation has also been shown for the frequency band from 250 to 800 Hz, for which the mean relative errors are within 2 dB. The average gap per frequency band is obtained by averaging the mean relative error spectrum over all 14 one-third octave bands. Following the same idea, the mean relative error in terms of overall sound pressure levels will also be used as a metric for model accuracy.

The average gap per frequency band and the mean relative error of overall sound pressure levels over all microphone positions are shown in tab. \ref{tab:average_gap}.


\begin{table}[H]
	\centering
	\caption{Average gap per frequency band and mean relative error of overall SPL of initial model}
	\begin{tabular}{c|c}
		Average gap per  frequency band (dB) & Mean relative error of overall SPL (dB) \\ \hline
		$2.14\pm1$                               & $0.73\pm0.86$                             
	\end{tabular}
	\label{tab:average_gap}
\end{table}

The average difference per frequency band is 2.14 dB and the mean relative error of overall sound pressure level is 0.73 dB. From these results it can be concluded that the finite element model is able to predict the sound transmission from the train underfloor adequately.


\section{Effect of geometric variation}

\begin{figure}[H]
	\centering
	\begin{subfigure}[b]{0.49\textwidth}
		\centering
		\includegraphics{fig/chap5/geometry_variation/third_octave_over_height/100_Hz.png}
		\caption{100 Hz}
	\end{subfigure}
	\begin{subfigure}[b]{0.49\textwidth}
		\centering
		\includegraphics{fig/chap5/geometry_variation/third_octave_over_height/250_Hz.png}
		\caption{250 Hz}
	\end{subfigure}
	\begin{subfigure}[b]{0.49\textwidth}
		\centering
		\includegraphics{fig/chap5/geometry_variation/third_octave_over_height/500_Hz.png}
		\caption{500 Hz}
	\end{subfigure}
	\begin{subfigure}[b]{0.49\textwidth}
		\centering
		\includegraphics{fig/chap5/geometry_variation/third_octave_over_height/1000_Hz.png}
		\caption{1000 Hz}
	\end{subfigure}
	\caption{Sound distribution at measurement position a, comparison between predictions obtained from the initial finite element model with the measurements. A-weighted SPL in one-third octave bands, dBA ref 20 $\mu$Pa.}
	\label{fig:third_octave_over_height_geometry_variation}
\end{figure}

\begin{figure}[H]
	\centering
	\begin{subfigure}[b]{\textwidth}
		\centering
		\includegraphics[width=0.49\linewidth]{fig/chap5/geometry_variation/overall_SPL/pos_a.png}
		\includegraphics[width=0.49\linewidth]{fig/chap5/geometry_variation/overall_SPL/pos_a_deviation.png}
		\caption{10 cm away from carbody}
	\end{subfigure}
	\begin{subfigure}[b]{\textwidth}
		\centering
		\includegraphics[width=0.49\linewidth]{fig/chap5/geometry_variation/overall_SPL/pos_f.png}
		\includegraphics[width=0.49\linewidth]{fig/chap5/geometry_variation/overall_SPL/pos_f_deviation.png}
		\caption{50 cm away from carbody}
	\end{subfigure}
	\begin{subfigure}[b]{\textwidth}
		\centering
		\includegraphics[width=0.49\linewidth]{fig/chap5/geometry_variation/overall_SPL/pos_g.png}
		\includegraphics[width=0.49\linewidth]{fig/chap5/geometry_variation/overall_SPL/pos_g_deviation.png}
		\caption{100 cm away from carbody}
	\end{subfigure}
	\caption{Comparison of overall SPL in dBA ref 20 $\mu$Pa. Left: overall SPL as a function of height at various measurement positions (dashed curves: measurement; solid curves: simulation) Right: deviation between simulation and measurement results}
	\label{fig:overall_SPL_geometry}
\end{figure}


\begin{figure}[H]
	\centering
	\includegraphics[width=0.7\linewidth]{fig/chap5/geometry_variation/freq_spectrum/average_gap.png}
	\caption{Mean relative error of 1/3-octave frequency}
	\label{fig:gap_freq_spectrum_geometry}
\end{figure}

\begin{table}[H]
	\centering
	\caption{Average gap per frequency band and mean relative error in overall SPL for different geometry variations}
	\begin{tabular}{c|c|c}
		Geometry name              & Average gap per frequency band (dB) & MRE overall SPL (dB) \\ \hline
		No underfloor components   & $3.12\pm1.74$                       & $1.78\pm0.91$        \\
		No air suspension          & $2.21\pm1.27$                       & $0.85\pm1.05$        \\
		Initial model              & $2.34\pm1.00$                       & $0.73\pm0.86$        \\
		With additional structures & $2.57\pm1.38$                       & $1.30\pm0.76$       
	\end{tabular}
\end{table}


\section{Effect of ground absorption}

\begin{figure}[H]
	\centering
	\begin{subfigure}[b]{\textwidth}
		\centering
		\includegraphics[width=0.49\textwidth]{fig/chap5/impedance/third_octave/SPL_100_Hz.png}
		\includegraphics[width=0.49\textwidth]{fig/chap5/impedance/third_octave/deviation_100_Hz.png}
		\caption{100 Hz}
	\end{subfigure}
	\begin{subfigure}[b]{\textwidth}
		\centering
		\includegraphics[width=0.49\textwidth]{fig/chap5/impedance/third_octave/SPL_1000_Hz.png}
		\includegraphics[width=0.49\textwidth]{fig/chap5/impedance/third_octave/deviation_1000_Hz.png}
		\caption{1000 Hz}
	\end{subfigure}
	\begin{subfigure}[b]{\textwidth}
		\centering
		\includegraphics[width=0.49\textwidth]{fig/chap5/impedance/third_octave/SPL_2000_Hz.png}
		\includegraphics[width=0.49\textwidth]{fig/chap5/impedance/third_octave/deviation_2000_Hz.png}
		\caption{2000 Hz}
	\end{subfigure}

	\caption{Sound distribution at measurement position a, comparison between predictions obtained from the initial finite element model with the measurements. A-weighted SPL in one-third octave bands, dBA ref 20 $\mu$Pa.}
	\label{fig:third_octave_over_height_impedance}
\end{figure}

\begin{figure}[H]
	\centering
	\begin{subfigure}[b]{\textwidth}
		\centering
		\includegraphics[width=0.49\textwidth]{fig/chap5/impedance/overall_SPL/overall_SPL_pos_a.png}
		\includegraphics[width=0.49\textwidth]{fig/chap5/impedance/overall_SPL/deviation_pos_a.png}
		\caption{10 cm away from carbody}
	\end{subfigure}
	\begin{subfigure}[b]{\textwidth}
		\centering
		\includegraphics[width=0.49\textwidth]{fig/chap5/impedance/overall_SPL/overall_SPL_pos_f.png}
		\includegraphics[width=0.49\textwidth]{fig/chap5/impedance/overall_SPL/deviation_pos_f.png}
		\caption{50 cm away from carbody}
	\end{subfigure}
	\begin{subfigure}[b]{\textwidth}
		\centering
		\includegraphics[width=0.49\textwidth]{fig/chap5/impedance/overall_SPL/overall_SPL_pos_g.png}
		\includegraphics[width=0.49\textwidth]{fig/chap5/impedance/overall_SPL/deviation_pos_g.png}
		\caption{100 cm away from carbody}
	\end{subfigure}
	
	\caption{Sound distribution at measurement position a, comparison between predictions obtained from the initial finite element model with the measurements. A-weighted SPL in one-third octave bands, dBA ref 20 $\mu$Pa.}
	\label{fig:overall_SPL_impedance}
\end{figure}

\begin{figure}[H]
	\centering
	\includegraphics[width=0.7\linewidth]{fig/chap5/impedance/freq_spectrum/average_gap.png}
	\caption{Mean relative error compared to full reflective model in 1/3-octave band}
	\label{fig:gap_freq_spectrum_impedance}
\end{figure}


\section{Effect of varing frequency steps per 1/3-octave band}

\begin{figure}[H]
	\centering
	\begin{subfigure}[b]{0.49\textwidth}
		\centering
		\includegraphics{fig/chap5/freq_steps/third_octave_over_height/100_Hz.png}
		\caption{100 Hz}
	\end{subfigure}
	\begin{subfigure}[b]{0.49\textwidth}
		\centering
		\includegraphics{fig/chap5/freq_steps/third_octave_over_height/315_Hz.png}
		\caption{315 Hz}
	\end{subfigure}
	\begin{subfigure}[b]{0.49\textwidth}
		\centering
		\includegraphics{fig/chap5/freq_steps/third_octave_over_height/630_Hz.png}
		\caption{630 Hz}
	\end{subfigure}
	\begin{subfigure}[b]{0.49\textwidth}
		\centering
		\includegraphics{fig/chap5/freq_steps/third_octave_over_height/2000_Hz.png}
		\caption{2000 Hz}
	\end{subfigure}
	
	\caption{Sound distribution at measurement position a, comparison between predictions obtained from the initial finite element model with the measurements. A-weighted SPL in one-third octave bands, dBA ref 20 $\mu$Pa.}
	\label{fig:third_octave_over_height_freq_steps}
\end{figure}


\begin{figure}[H]
	\centering
	\begin{subfigure}[b]{0.3\textwidth}
		\centering
		\includegraphics[width=\linewidth]{fig/chap5/freq_steps/field_result_1781Hz.png}
		\caption{1781 Hz}
	\end{subfigure}
	\hfill
	\begin{subfigure}[b]{0.3\textwidth}
		\centering
		\includegraphics[width=\linewidth]{fig/chap5/freq_steps/field_result_2000Hz.png}
		\caption{2000 Hz}
	\end{subfigure}
	\hfill
	\begin{subfigure}[b]{0.3\textwidth}
		\centering
		\includegraphics[width=\linewidth]{fig/chap5/freq_steps/field_result_2245Hz.png}
		\caption{2245 Hz}
	\end{subfigure}
	\caption{Pressure field of single frequency}
\end{figure}
