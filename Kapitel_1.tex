\chapter{Grundlegende \LaTeX-Befehle}
\label{chap:Grundlegendes}

Dieses Kapitel stellt die wichtigsten Schritte und Befehle für die Erstellung eines
Dokumentes vor. 

Das Erzeugen von Postscript-Dokumenten erfolgt in drei Schritten:
\begin{enumerate}
\item Erstellen des *.tex-Files mit einem Texteditor,
\item Kompilieren des *.tex-Files in ein dvi-File mit {\sf latex
    filename.tex}
 ("Uberpr"ufen des Layouts mit z.B. {\sf kdvi} oder {\sf xdvi}),
\item Konvertierung des dvi-Files in ein *.ps-File f"ur den Ausdruck
  mit {\sf dvips filename}
\end{enumerate}

Soll ein pdf-Dokument generiert werden, sind es nur noch zwei
Schritte. Mit  {\sf pdflatex filename.tex} wird direkt die Datei
filename.pdf erzeugt.


\section{Editoren und Previewer}
\label{sec:Editor}

Unter Linux bietet sich der Xemacs als Editor an. Andere Editoren
(Kwrite, Nedit) unterst"uzten die Syntax-Hervorhebung, erm"oglichen jedoch
nicht das Einf"ugen von z.B. {\it figure}-Umgebungen. Unter Windows
gibt es auch einen Xemacs (Freeware). Zu empfehlen ist der Editor Texmaker. Ein weiterer Editor, der WinEdt (Shareware),
integriert die oben genannten Schritte und einen Previewer (Yap) in eine Umgebung.  

Als dvi-Previewer stehen unter Linux xdvi und kdvi zur Verf"ugung.
Der kdvi-Previewer hat den großen Vorteil, daß eingebundene Postscript-Bilder nicht angezeigt werden (nur die Umrisse), wodurch die Anzeige einer neuen Seite wesentlich schneller als beim xdvi-Previewer 
durchgeführt wird! Soll jedoch auch der Inhalt eines Bildes angezeigt werden, muß der xdvi-Previewer 
mit dem Nachteil eines sehr langsamen Seitenaufbaues verwendet werden.  
Durch das Dokument kann  
mit PageUp bzw. PageDown geblättert werden. Im kdvi-Previewer kann mit
der linken Maustaste die angezeigte  Fläche einer Seite verändert
werden.

Ein Previewer f"ur Windows ist z.B. der schon oben genannte Yap.

Die einfachste Art um zu einer pdf-Datei zu gelangen ist das direkte kompilieren mit Hilfe von pdflatex (Dies kann bspw. bei Texmaker unter Optionen - Texmaker konfigurieren - Schnelles Übersetzen - PdfLaTeX + PDF anzeigen eingestellt und mit der F1-Taste ausgeführt werden).



\section{Das Zentral- und die Filialdokumente}
Das {\it Zentraldokument} beginnt mit der Angabe des Dokument- und
Seitenstiles.  Durch die Angabe des {\it include} bzw.
{\it includeonly}-Befehles werden die {\it Filialdokumente} eingebunden. Das Kompilieren muß
stets aus dem Zentraldokument erfolgen. Die Struktur des Zentraldokumentes ist
wie folgt:

{\footnotesize
\begin{verbatim}
\documentclass{article}

\usepackage{...}  % Einbinden von packages

\begin{document}

%\includeonly{zeichen,kap_1}  % zur gezielten Bearbeitung eines Kapitels

\pagenumbering{roman}
\include{title}  % Titelblatt der Diplomarbeit
\include{vorwort}  % Vorwort hier einbinden, falls erwünscht

\setcounter{page}{1}
\tableofcontents   % Inhaltsverzeichnis

\include{zeichen}  % Verwendete Symbole und Größen

\pagestyle{headings}
\pagenumbering{arabic}

\include{einleitung}  % Motivation und Umfang der Arbeit
\include{kapitel1}   % 1.tes Kapitel
\include{zusammen} % Zusammenfassung
\include{literatur}  % Literaturverzeichnis
\end{document}

\end{verbatim}
}

Mit dem Befehl {\sf latex zentraldokument} wird ein dvi-File
erzeugt. Daraus kann mit {\sf dvips} ein PostScript-File generiert
werden. Wird ein pdf-File ben"otigt, so geht dies mit {\sf pdflatex zentraldokument}.   

\section{Gleichungen und Formeln}
Hier bietet \LaTeX\,sehr umfangreiche Möglichkeiten.  Es seien hier nur einige
Beispiel angeführt.  
\\

\begin{eqnarray}
\bar{T}^h(t) &=& \sum_{i=1}^{n_{\rm eq}}
N_i(\vec{r})\bar{T}_i(t)\\[2mm]
W^h &=& \sum_{i=1}^{n_{\rm eq}} N_i(\vec{r})c_i\\[2mm]
T_{\rm e}^h(t) &=& \sum_{i=1}^{n_{\rm e}} N_i(\vec{r})T_{{\rm e}i}(t)
\label{equ:roman}
\end{eqnarray}

\begin{equation}
\vec{x}\left(\vec{\xi}\right)\,=\,
\left( \begin{array}{c}
x(\xi,\eta)\\
y(\xi,\eta) \end{array} \right)
\,=\,\sum_{i=1}^{4}
\left( \begin{array}{c}
N_i(\xi,\eta)x_i^e \\
N_i(\xi,\eta)y_i^e \end{array} \right)
\label{equ:gllokKoord}
\end{equation}

\begin{eqnarray}
 x(\xi_i,\eta_i)&=&x_i^e \nonumber\\
 y(\xi_i,\eta_i)&=&y_i^e
\end{eqnarray}

\begin{equation}
\int\limits_{\Omega} f(x,y)\,dxdy
\label{equ:IntvTrans}
\end{equation}

\begin{equation}
{\bf K}=\left(
\begin{array}{c c c c}
K_{11} & K_{12} & \cdots & K_{1n} \\
K_{21} & K_{22} & \cdots & K_{2n} \\
\vdots & \vdots & \ddots & \vdots \\
K_{n1} & K_{n2} & \cdots & K_{nn} 
\end{array}
\right)
\end{equation}


Wenn Sie eine Formel in einen Satz einbinden, so geben
Sie die erforderlichen Satzzeichen am Ende der Formel mit an.  Beachten Sie
auch die diversen Formatierungsunterscheidungen in den Beispielen wie z.B.
die {\it italic}-Schreibweise laufender Indizes im Gegensatz zur
{\it roman}-Schreibweise fester Indizes, siehe Gl. (\ref{equ:roman}).

Bei der Referenzierung von Gleichungen wird das Wort Gleichung nur am
Satzanfang ausgeschrieben, ansonsten abgekürzt.  Selbiges gilt auch für
Abbildungen und Abschnittsreferenzen.  Die große Ausnahme von dieser Regel
sind jedoch die Tabellen, diese werden immer ausgeschrieben.


\section{Die Grafiken}
\label{sec:Grafik}

Grafiken werden mit {\sf \textbackslash includegraphics} innerhalb einer {\it
  figure}-Umgebung eingebunden. Die Endung (.eps oder .png) sollte
dabei weggelassen werden. \LaTeX { }bzw. pdf\LaTeX { }suchen selbstst"andig nach den
richtigen Dateien. 
Das unterstützte Grafikformat ist Postscript (*.eps), wenn ein
DVI-File erzeugt wird und das JPG-Format (nur Linux/Unix) bzw. das
PNG-Format (Linux/Unix und Windows) f"ur PDF als Zielformat. Zwecks
Portabilit"at ist es sinnfoll nur *.eps und *.png Grafiken zu
verwenden. Diese
Grafikdateien sollten einfach im gleichen Verzeichnis mit den
*.tex-Dateien liegen. Alternativ kann auch ein anderes Verzeichniss im
Kopf des Zentraldokumentes angegeben werden (mit {\sf \textbackslash graphicspath\{\{./fig/\}\} }).

% Beispiel für eine Grafik
\begin{figure}[htb]
\begin{center}
\includegraphics[width=10cm]{fig/beispiel_1}
\caption{Diagnose und HIFU-Therapie mit einem Array.}
\label{fig:Beispiel1}
\end{center}
\end{figure}

Die Befehle für die Einbindung der Grafik in ein \TeX-File sind anhand
der {\it figure}-Umgebung der Abb. \ref{fig:Beispiel1} zu
erkennen. Abbildung \ref{fig:lbp} zeigt ein Beispiel mit zwei Bildern
nebeneinander. Sollen zwei (oder mehr) Bilder in einer {\it 
figure}-Umgebung mit einzelnen Bildunterschrifften versehen werden, so
geht dies mit dem {\sf \textbackslash subfigure-Befehl}. Abbildung
\ref{fig:subfigure} zeigt daf"ur ein Beispiel.

\begin{figure}[hbt]
\begin{center}
   \begin{minipage}[t]{7.45cm}
     \includegraphics[width=7cm]{fig/lb}
   \end{minipage}
   \hfill
   \begin{minipage}[t]{7.45cm}
     \includegraphics [width=7cm]{fig/lp}
   \end{minipage}
   \caption[Dieser Teil steht in der Liste der Abbildungen (siehe Quelltext).]
           {Dieser Teil steht unter dem Bild als Erkl"arung (siehe Quelltext).}
   \label{fig:lbp}
\end{center}
\end{figure}


\begin{figure}[htbp]

  \begin{center}
      \subfigure[Dispersionskurven der drei Lamb-Wellen niedrigster Ordnung (nach {\changefont \cite{messtechnik})}.]{
      \includegraphics[width=6.5cm]{fig/lp}
      \label{fig:auldLambDisper}
      }
    \hspace{0.5cm}
    \subfigure[3D-Darstellung der simulierten Dispersionskurven einer
    Platte.]{    
      \includegraphics[width=6.5cm]{fig/lb}
      \label{fig:feLambDisper}  
      }  
    \caption[Vergleich analytische L"osung und Simulation.]{Qualitativer Vergleich zwischen
analytischer Lösung und Simulation mit Hilfe periodischer
Randbedingungen (Beispiel mit {\sf subfigure} und zus"atzlichen getrennten Bildunterschriften).}
    \label{fig:subfigure}
  \end{center}
\end{figure}


Anhand der hier eingebundenen Grafiken ist auch gleichzeitig die Bedeutung
der {\it float}-Umgebung zu erkennen.  Aufgrund einer allgemeinen Konvention
versucht \TeX\, selbstständig eine möglichst günstige Aufteilung der Grafiken
zu finden.  Wundern Sie sich daher nicht, wenn Ihre Grafik einmal nicht dort
zu finden ist, wo Sie sie eingebunden haben. Die Abbildung schwimmt
(floats) im Text an eine g"unstige Stelle.


Alle Abbildungen (sowie Tabellen und Gleichungen) sollten mit einem
Label versehen werden. Innerhalb der {\it figure}-Umgebung wird ein
{\sf \textbackslash label\{fig:PrettyPicture\}} gesetzt.  Im Text wird mit dann mit
{\sf \textbackslash ref\{fig:PrettyPicture\}} darauf verwiesen. 



\section{Tabellen und Aufzählungen}
\label{sec:Tabellen}

Um Gliederungen, Tabellen und Aufzählungen zu realisieren gibt es
verschiedene Möglichkeiten:
\begin{itemize}
\item {\it itemize}-Umgebung (diese hier)
\item {\it tabular}-Umgebung
\item {\it tabbing}-Umgebung
\item {\it enumerate}-Umgebung (siehe Beginn Kapitel \ref{chap:Grundlegendes}) 
\end{itemize}

Gr"o"sere Tabellen kommen sinnvollerweise in eine {\it table}-
Umgebung. Diese schwimmen wie die {\it figure}-Umgebungen im Text
(Beispiel: Tabelle \ref{tab:Beispiel} und
\ref{tab:Parameter}). Tabelle \ref{tab:Parameter} zeigt auch
M"oglichkeiten einzelne Tabellenzellen zu verbinden.


\begin{table}[hbt]
\begin{center}
\caption[Unterschiedliche Materialdaten.]{Unterschiedliche Materialdaten einer
         POCO-Schicht bei unterschiedlichem Vorgehen beim Tränken.} \vskip1ex
 \label{tab:Beispiel}
  \begin{tabular}{lll} \hline\\[-4mm]
  Verfahren                                 & Dichte     & Wellengeschwindigkeit\\
                                            & kg/m$^3$   & m/s \\[1mm]\hline\\[-3mm]
  Tränken einer $10\,{\rm mm}$ POCO-Schicht & 1758       & 3030\\
  Tränken einer $150\,{\rm \mu m}$ POCO-Schicht & 1920   & 3250\\
  Längeres Tränken einer $150\,{\rm \mu m}$ POCO-Schicht & 1940 & 3410\\[1mm]\hline
  \end{tabular}
\end{center}
\end{table}

\begin{table}[hbt]
\begin{center}
\caption{Parameter für kennlinienkorrigierten PID-Regler in der
  Simulation.} \vskip1ex
\label{tab:Parameter}
\setlength{\doublerulesep}{0.2pt} % zwei Linien erscheinen als eine dicke
\begin{tabular}{|c|c|c|c|c|c|c|} \hline
& \multicolumn{6}{c|}{\emph{Parameter}} \\ \cline{2-7}
\raisebox{1.5ex}[-1.5ex]{\emph{Sollverlauf}} & $k_p$ & $T_d$ in s & $T_i$ in s & $w_d$ & $N_d$ & $g$ \\ \hline\hline
sinusförmig   & -150 & 0.02  & 0.2 & 0.2 & 500 & 0.2  \\  \hline
trapezförmig  & -200 & 0.03  & 0.4 & 0.2 & 500 & 0.2  \\  \hline
treppenförmig & -150 & 0.025 & 0.2 & 0.2 & 500 & 0.2  \\  \hline
\end{tabular}
\end{center}
\end{table}


Die Tabelle der verwendeten Formelzeichen (siehe dort) wurde mittels einer
{\it tabbing}-Umgebung erstellt.
