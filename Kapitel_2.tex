\section{Beispielausschnitt}

Zum Abschluß noch ein kurzer Ausschnitt aus einer Diplomarbeit.

\subsection{Bilineares Rechteckselement \label{Bilinear}}

\begin{figure}[tb]
\begin{center}
\includegraphics[width=9cm]{fig/pic_4_1}
\caption{Zusammenhang: globales, lokales Koordinatensystem.}
\label{fig:KoordSysteme}
\end{center}
\end{figure}

Für den Zusammenhang zwischen globalem und lokalem Koordinatensystem (\ref{fig:KoordSysteme}) gilt

\begin{equation}
\vec{x}\left(\vec{\xi}\right)\,=\,
\left( \begin{array}{c}
x(\xi,\eta)\\
y(\xi,\eta) \end{array} \right)
\,=\,\sum_{i=1}^{4}
\left( \begin{array}{c}
N_i(\xi,\eta)x_i^e \\
N_i(\xi,\eta)y_i^e \end{array} \right)\,.
%\label{gllokKoord}
\end{equation}

Man verwendet den bilinearen Ansatz

\begin{eqnarray}
x(\xi,\eta)&=&\alpha_0\,+\,\alpha_1\xi\,+\,\alpha_2\eta\,+\,\alpha_3\xi\eta
\nonumber\\
y(\xi,\eta)&=&\beta_0\,+\,\beta_1\xi\,+\,\beta_2\eta\,+\,\beta_3\xi\eta\,.
\end{eqnarray}

Für jeden Eckpunkt $(x_i^e,y_i^e)$ eingesetzt, erhält man die
Bestimmungsgleichungen für \linebreak $\alpha_0,\alpha_1,\alpha_2,\alpha_3$
und $ \beta_0,\beta_1,\beta_2,\beta_3  $ aus

\begin{eqnarray}
 x(\xi_i,\eta_i)&=&x_i^e \nonumber\\
 y(\xi_i,\eta_i)&=&y_i^e\,.
\end{eqnarray}

Nach Einsetzen der Lösung und Koeffizientenvergleich mit obiger Gleichung
erhält man

\begin{equation}
N_i(\xi,\eta)\,=\,\frac{1}{4}\left(1+\xi_i\xi\right)\left(1+\eta_i\eta\right)\,.
\end{equation}

\begin{figure}[htb]
\begin{center}
\includegraphics[width=4cm]{fig/pic_4_2}
\caption{Ansatzfunktion für den i-ten Knoten.}
\label{fig:Ansatzfunktion}
\end{center}
\end{figure}

Das Element heißt isoparametrisch, wenn die selbe Ansatzfunktion $N_i$ für die
Koordinatentransformation und die Lösungsfunktion verwendet wird
(siehe auch \ref{fig:Ansatzfunktion}).

\subsection{Ableitung der Formfunktion}

Betrachtet wird die {\em semidiskretisierte Galerkin-Formulierung}.
Für die Ermittlung der Leitfähigkeitsmatrix $\bf K$ mit

\begin{eqnarray}
K_{ij} &=& \int\limits_\Omega \left(\frac{\partial N_i}{\partial x}k\frac{\partial
N_j}{\partial x}\,+\,\frac{\partial     N_i}{\partial y}k\frac{\partial
N_j}{\partial y}\right)\,dxdy\,+\,\int\limits_{\Gamma_{\rm c}}N_i\alpha N_j\,d\Gamma
\nonumber \\[2mm]
&&1 \le \,i,j\, \le n_{\rm eq}
\end{eqnarray}

ist es erforderlich, die Ansatzfunktion $N_i$ nach den globalen Koordinaten
$(x,y)$ abzuleiten.  Bei der Verwendung eines lokalen Koordinatensystems mit
den Koordinaten $(\xi,\eta)$ wird die Ableitung
wie folgt berechnet

\begin{equation}
\frac{\partial N_i}{\partial x}\,=\,\frac{\partial N_i}{\partial \xi}
\frac{\partial \xi}{\partial x}\,+\,\frac{\partial N_i}{\partial \eta}
\frac{\partial \eta}{\partial x}\,.
\end{equation}

Die Bestimmung der Funktion     $ \vec{\xi}(\vec{x}) $ ist mit
erheblichem Aufwand verbunden. Da die Funktion $ \vec{x}(\vec{\xi}) $
jedoch bekannt ist, hilft man sich auf folgende Weise

\begin{eqnarray}
\frac{\partial N_i}{\partial \xi}\,=\,\frac{\partial N_i}{\partial x}
\frac{\partial x}{\partial \xi}\,+\,\frac{\partial N_i}{\partial y}
\frac{\partial y}{\partial \xi} \nonumber \\[2mm]
\frac{\partial N_i}{\partial \eta}\,=\,\frac{\partial N_i}{\partial x}
\frac{\partial x}{\partial \eta}\,+\,\frac{\partial N_i}{\partial y}
\frac{\partial y}{\partial \eta}\,.
\end{eqnarray}

Somit lassen sich die Ableitungen der Ansatzfunktion nach den globalen
Koordinaten mit Hilfe der Jakobi Matrix $\bf J$

\begin{equation}
{\bf J}\,=\,
\left(
\begin{array}{cc}
\displaystyle\frac{\partial x}{\partial \xi} &
\displaystyle\frac{\partial y}{\partial \xi} \\[4mm]
\displaystyle\frac{\partial x}{\partial \eta} &
\displaystyle\frac{\partial y}{\partial \eta}
\end{array} \right)
\end{equation}

wie folgt darstellen

\begin{equation}
\left( \begin{array}{c} \displaystyle\frac{\partial N_i}
{\partial x} \\[4mm]
\displaystyle\frac{\partial N_i}{\partial y}
\end{array} \right)\,=\,{\bf J}^{-1}
\left( \begin{array}{c} \displaystyle\frac{\partial N_i}
{\partial \xi} \\[4mm]
\displaystyle\frac{\partial N_i}{\partial \eta}
\end{array} \right)\,.
\end{equation}


