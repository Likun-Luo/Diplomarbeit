\chapter{Introduction}

Im Gegensatz zu den Textverarbeitungsprogrammen wie WinWord und Konsorten
stellt \LaTeX\, einen Textcompiler dar.  Es ist somit möglich mit jedem
beliebigen Editor ein \LaTeX -File zu erstellen.  Erst nach erfolgter
Compilierung kann mit Hilfe des {\it Previewers} die formatierte Seite
betrachtet werden.

Die hier gegebene Einführung ist weniger als \LaTeX -Anleitung zu verstehen
als vielmehr ein Überblick über die am Institut vorhandene
Entwicklungsumgebung.  Weiters soll Ihnen ein Grundkonzept für die Erstellung
der Dokumentation Ihrer Arbeit und eine Grobgliederung gegeben werden.
Wenn Sie spezielle \LaTeX -Befehle suchen, ist am Institut ausreichend
Literatur (z.B. {\changefont \cite{Texbook}}) vorhanden, oder fragen Sie ihren Betreuer.

\bigskip

Um eine sinnvolle und übersichtliche Dokumentation Ihrer Arbeit schreiben zu
können ist es unerläßlich, ein ausgereiftes Gliederungskonzept zu haben.
Denken Sie daran, daß die Arbeit auch für jemanden verständlich sein soll,
der mit der behandelten Materie nicht so sehr vertraut ist wie Sie, außerdem
sollte man in der Lage sein, sich in kurzer Zeit einen Überblick über die
behandelten Gebiete zu verschaffen.  Aus diesem Grund ist eine Einleitung
erforderlich in der kurz die Aufgabenstellung und die Lösungsansätze
beschrieben werden sollten.  Weiters sollten Ihre Ergebnisse und Folgerungen
überblicksartig in einer Zusammenfassung wiederholt werden.

Teilen Sie die Arbeit so auf, daß Sie dem Leser die Möglichkeit geben, sich in
die Arbeit einzulesen, erschlagen Sie ihn nicht gleich mit hochtheoretischen
Betrachtungen oder detailierten Schaltungsdimensionierungen.  Oft ist eine
Erläuterung der Arbeit anhand eines allgemein gehaltenen Blockschaltbildes
sehr nützlich, um dem Leser die Zusammenhänge zu verdeutlichen.


% cite siehe Literaturverzeichnis

